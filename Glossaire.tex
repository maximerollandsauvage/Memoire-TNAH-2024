\newglossaryentry{LNR}{
    name={listes nominatives de recensement},
    description={ Les listes nominatives de recensement sont des documents temporaires, produits à intervalles réguliers, généralement tous les cinq ou dix ans, pour recenser la population d'une commune ou d'une région à un moment donné. Ces listes contiennent des informations comme le nom, l'âge, la profession, et le lieu de résidence des individus, mais elles ne documentent pas des événements de la vie, plutôt un état de la population à un instant T. Elles sont principalement utilisées à des fins statistiques ou pour des études démographiques}
}

\newglossaryentry{AEC}{
    name={actes d'état civil},
    description={Les actes d'état civil, tels que les actes de naissance, de mariage et de décès, sont des documents officiels qui enregistrent des événements spécifiques de la vie d'une personne. Ils sont rédigés par l'officier d'état civil de la commune où l'événement a eu lieu et ont une valeur juridique. Ces actes sont permanents et servent de preuve légale de l’identité et de l’état civil des individus}
}

\newglossaryentry{entités}{
    name={entités},
    description={Dans le cadre d'une base de données, une entité est un objet concret ou abstrait qui possède des caractéristiques (appelées "attributs") et sur lequel des informations sont enregistrées}
}

\newglossaryentry{DL}{
    name={deep learning},
    description={Le deep learning, ou apprentissage profond, est une sous-catégorie du machine learning qui se base sur des réseaux de neurones artificiels à plusieurs couches, souvent appelés réseaux de neurones profonds. Ces réseaux sont capables de modéliser des représentations complexes en passant par plusieurs niveaux d'abstraction, chaque couche du réseau transformant les données de manière non linéaire. Le deep learning est particulièrement efficace pour traiter des données volumineuses et complexes, comme les images, le son, et le langage naturel, et est utilisé dans des applications telles que la reconnaissance vocale, la vision par ordinateur, et les véhicules autonomes. Le deep learning est considéré comme une évolution du machine learning}
}

\newglossaryentry{ML}{
    name={machine learning},
    description={Le machine learning, ou apprentissage automatique, est une branche de l'intelligence artificielle (IA) qui consiste à créer des algorithmes capables d'apprendre à partir de données. Plutôt que de suivre des instructions codées manuellement, ces algorithmes identifient des motifs et prennent des décisions basées sur des exemples fournis lors de la phase d'apprentissage. Une définition claire des entités est cruciale pour la conception de la base de données, car elle détermine la manière dont les données seront structurées, interconnectées et accessibles}
}

\newglossaryentry{océrisation}{
    name={océrisation},
    description={L'OCR est une technologie utilisée pour reconnaître et extraire du texte à partir de documents imprimés, tels que des livres, des articles, des factures, ou tout autre type de document contenant du texte typographié ou dactylographié. Les algorithmes OCR sont optimisés pour traiter des caractères qui suivent des polices standardisées et régulières, ce qui les rend très efficaces pour convertir du texte imprimé en texte numérique}
}

\newglossaryentry{HTR}{
    name={HTR},
    description={L'HTR, quant à elle, est une technologie plus spécialisée, conçue pour reconnaître et extraire du texte manuscrit. Contrairement à l'OCR, l'HTR doit gérer les variations naturelles dans l'écriture manuscrite, comme les différentes formes de lettres, les styles d'écriture, et parfois les imperfections ou irrégularités dans les documents manuscrits. L'HTR est utilisée pour numériser et convertir des textes écrits à la main, comme des lettres, des notes, des registres historiques, ou tout autre document manuscrit. L'HTR est généralement plus complexe à mettre en œuvre en raison de la variabilité plus grande des formes de caractères dans les textes manuscrits}
}

\newglossaryentry{IIIF}{
    name={IIIF},
    description={La technologie IIIF (International Image Interoperability Framework) fonctionne via deux principales API : l'Image API et la Presentation API. L’image API permet d'accéder à une image source en spécifiant des paramètres tels que la taille, le recadrage, la rotation, ou la qualité. Cela permet de demander exactement la portion d'image souhaitée, directement depuis un serveur compatible IIIF. La présentation API organise et décrit les images en leur associant des métadonnées et une structure (comme une séquence de pages pour un manuscrit). Cela permet aux visionneuses de présenter les images avec un contexte structuré, facilitant leur exploration. Ces API utilisent des URL normalisées pour permettre une interopérabilité et une flexibilité accrues entre différents systèmes et plateformes}
}

\newglossaryentry{households}{
    name={households},
    description={Un foyer est une entité résidentielle comprenant une ou plusieurs personnes vivant ensemble à une adresse donnée, qui peut être un domicile familial, un appartement, ou toute autre forme de logement. Lors d'un recensement, chaque foyer est enregistré pour collecter des informations sur ses membres, telles que leur nombre, leur âge, leur sexe, leur situation professionnelle, et d'autres caractéristiques démographiques}
}

\newglossaryentry{ARK}{
    name={ARK},
    description={Le chemin ARK (Archival Resource Key) est une URL persistante utilisée pour identifier de manière stable et unique des objets numériques, tels que des documents, images, ou autres ressources archivées, sur le web. Il garantit que les liens vers ces ressources restent valides sur le long terme, même si leur emplacement physique ou leur gestion change}
}

\newglossaryentry{SP}{
    name={single page},
    description={Dans le contexte de la Reconnaissance de Texte Manuscrit (HTR), les termes single page et double page se réfèrent à la manière dont les pages d'un document manuscrit ou imprimé sont traitées pour l'analyse et la conversion en texte numérique. Le traitement en single page se fait sur chaque page séparément, tandis que pour les double page, deux pages côte à côte sont gérés comme une seule entité pour capturer le texte de manière continue}
}

\newglossaryentry{AGILE}{
    name={AGILE},
    description={La méthode agile est une approche de gestion de projet, principalement utilisée dans le développement logiciel, qui se caractérise par sa flexibilité et son adaptabilité. Contrairement aux méthodes traditionnelles qui suivent un plan rigide, l'agile favorise des cycles de développement courts appelés "sprints", au cours desquels des fonctionnalités du produit sont développées, testées, et améliorées en continu. La collaboration étroite entre les équipes et les parties prenantes, ainsi qu'une réactivité rapide aux changements, sont au cœur de cette méthode. L'objectif est de livrer un produit fonctionnel régulièrement, tout en s'adaptant aux besoins évolutifs du projet et des utilisateurs}
}

\newglossaryentry{SQL}{
    name={SQL},
    description={SQL (Structured Query Language) est un langage de programmation standardisé utilisé pour gérer et manipuler des bases de données relationnelles. Il permet de réaliser diverses opérations telles que la création, la modification, la suppression et la requête de données au sein des bases de données. SQL est essentiel pour interagir avec les systèmes de gestion de bases de données (SGBD) comme MySQL, PostgreSQL, Oracle, ou Microsoft SQL Server. Il est largement utilisé pour récupérer des informations spécifiques à partir de grandes quantités de données, gérer la structure des bases de données, et contrôler l'accès aux données.}
}


    \newacronym{INED}{INED}{Institut National des Etudes Démographiques}
    \newacronym{SIAF}{SIAF}{Service Interministériel des Archives Françaises}
    \newacronym{PSE}{PSE}{Paris Schoolf of Business}
    \newacronym{OCR}{OCR}{Optical Character Recognition}
    \newacronym{DAN}{DAN}{Document Attention Network}
    \newacronym{YOLO}{YOLO}{You Only Look Once}
    \newacronym{PSL}{PSL}{Paris Sciences et Lettres}
    \newacronym{EHESS}{EHESS}{Ecole des Hautes Etudes en Sciences Sociales}
    \newacronym{ENS}{ENS}{Ecole Normale Supérieure}
    \newacronym{LITIS}{LITIS}{Laboratoire d'Informatique, de Traitement de l'Information et des Systèmes}

