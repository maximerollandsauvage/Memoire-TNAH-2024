\chapter{Entre interdisciplinarité et collaboration}
Le projet SocFace est porté par plusieurs acteurs : l'\gls{INED}, PSE, le SIAF, et Teklia.Au-delà de la diversité de ces institutions, de leurs statuts et de leurs missions, ce projet illustre l’importance de l’interdisciplinarité dans la recherche. Plus spécifiquement, pour SocFace, il est intéressant d’étudier la dynamique entre des institutions de la recherche publique coopérant avec Teklia, entreprise privée. 

    \section{Les acteurs du projet}

Le projet SocFace implique plusieurs acteurs aux rôles divers. Les deux principales entités dirigeantes sont l'{\gls{INED}} et Teklia, tandis que le \gls{SIAF} et \gls{PSE} interviennent à certaines étapes du projet.\\
l'\gls{INED} est un Établissement Public à Caractère Scientifique et Technologique (EPST). Les sujets couverts par les chercheurs sont très larges : démographie, genre, sexualité, migrations, familles etc… Ces chercheurs ont des profils très divers : démographes, statisticiens, ingénieurs informatiques, sociologues ou historiens. Le projet est initié par Lionel Kesztenbaum, responsable de l’unité Histoire et Populations. Il n’y a pas d’équipe dédiée au projet à temps plein, mais une doctorante, un chercheur associé et parfois des stagiaires. D’autres chercheurs interviennent ponctuellement. Les équipes de l'\gls{INED} apportent leur expertise en sciences sociales. Il faut cependant noter que la plupart ont aussi des compétences en traitement et gestion de bases de données.\\
Teklia est une société fondée en 2014 par Christopher Kermorvant. Elle se spécialise sur le traitement automatique en langage naturel et la reconnaissance optique de caractère, ce qui est à la base du projet SocFace. En utilisant la technologie du \gls{DL} et de l’intelligence artificielle, Teklia permet la conversion de documents en données structurées, interrogeables dans une base de données. Ces solutions s’adressent à des entreprises, des industriels ou institutions publiques. Le projet SocFace s’inscrit dans leurs projets en Recherche et Développement (R\&D). SocFace n’est d’ailleurs qu’un projet parmi d’autres, car Teklia s’est associé à de nombreux partenaires publics dans la sauvegarde du patrimoine public. On peut citer les Archives Nationales, l’Institut de la Recherche et d’Histoire des Textes (IRHT), ou la National Norwegian Library. Les équipes de Teklia dédiées à SocFace sont composées d’ingénieurs informatiques, ingénieurs réseaux et architectes, qui mettent au point les logiciels et programmes nécessaires à la réalisation du projet. 
Paris School of Economics est une institution de recherche fondée en 2006 par le CNRS, l’école des Ponts, \gls{PSL}, Paris 1, l'\gls{ENS} et l’\gls{EHESS}. L’objectif est de créer une communauté de chercheurs et d’enseignants au service des questions économiques, dans différents domaines. C’est un établissement qui prône l‘interdisciplinarité et cherche à faire travailler ensemble chercheurs en sciences sociales et économiques ou ingénieurs. L’un des objectifs de SocFace est justement de pouvoir améliorer la recherche historique et économique de la France sur la période. L’un des chercheurs associé au projet, Jérôme Bourdieu, a d’ailleurs déjà participé à une précédente enquête avec Lionel Kesztenbaum : L’enquête TRA, qui s’intéressait déjà la population française au XIXe siècle.\\
Enfin, le projet est porté par le SIAF. Ce service permet la coordination des différentes archives municipales, départementales et nationales. Il est chargé de faire appliquer des politiques communes concernant les archives. Concernant SocFace, c’est bien l'\gls{INED} et Teklia qui contactent ces services mais le \gls{SIAF} a permis la publicisation au début du projet. Il intervient en cas de blocage ou de lenteur dans les échanges et est un point de contact essentiel pour les services d'archives. La base de données pourra être consultée par ceux qui viennent consulter les Archives, faisant gagner un temps considérable aux usagers. Les services d’archives ont donc tout intérêt à participer au projet, mais cela peut parfois représenter un surcroit de travail important. De fait, tous les services n’avaient pas nécessairement numérisé les listes de recensement avant le début du projet.\\
Ajoutons que le projet est financé par l’ANR, l’Agence Nationale pour la Recherche, ce qui est bien entendu un gage de qualité, mais démontre aussi que le projet a dû respecter un certain nombre de conditions et est fortement attendu dans ses résultats.\\

Ainsi, nous avons vu qui étaient les équipes derrière le projet SocFace. Leur parcours dans la recherche diffère car il y a d’un côté des instituts de recherche classique, comme l'\gls{INED} et \gls{PSE}, et de l’autre des structures pour qui la recherche est seulement une partie de leurs activités. Mais une autre dichotomie traverse ces équipes : le secteur public collabore ici avec une entreprise privée. 
    
    \section{Les avantages du partenariat entre le public et le privé}

De fait, comme beaucoup de projet de recherche, SocFace associe public et privé. Ce type de partenariats n’est pas rare, même si on le retrouve davantage dans les projets de sciences dures comme la médecine ou la chimie. Selon la publication \textit{État de l’Enseignement supérieur, de la Recherche et de l’Innovation en France n°12}, on distingue trois types de partenariats dans la recherche : 

    \begin{itemize}[label=\textbullet] % Change les puces en points noirs
    \item \textbf{La recherche contractuelle},qui implique un commanditaire privé sous-traitant des travaux de Recherche \& Développement (R\&D) à une université ou un laboratoire. Cela représente 5,2\% de la recherche
    \item \textbf{La recherche collaborative}, qui permet l’association d’une entreprise privée avec un laboratoire et dans laquelle les coûts et les résultats sont partagés. 
    \item \textbf{Les travaux de consultance} dans lesquels un.e chercheur.se partage sa parole d’expert.e auprès d’une entreprise privée. 
\end{itemize}

En l’espèce, les acteurs de SocFace se situent dans la deuxième catégorie : les acteurs sont à égalité dans la responsabilité du projet et travaillent ensemble au quotidien. Ce type de recherche est choisi par 17\% des entreprises technologiquement innovantes, et le chiffre est en hausse. Elle est également davantage prisée par les grandes entreprises (plus de 250 salariés), ce qui n’est pas le cas de Teklia. De même, les entreprises coopèrent davantage avec les universités ou les établissements d’enseignements supérieurs. En l’espèce,l’\gls{INED} et \gls{PSE} ne sont pas des établissements d'enseignement supérieurs mais des instituts de recherche. Ce cas de figure représente 11\% des configurations. SocFace n’est donc pas un projet typique des partenariats publics/privés, mais s’inscrit malgré tout dans une tendance grandissante de la recherche, qui encourage les partenariats.

On ne parle pas ici de PPP (Partenariats Public-Privé) qui désigne plus généralement des marchés de partenariats, particulièrement dans le secteur hospitalier ou de la construction. Mais il est intéressant de comprendre la dynamique de travail entre une entreprise privée et des institutions publiques - en dehors de tout cadre administratif. Dans un article intitulé \textit{Le concept de "partenariat public" est-il bien posé ? la coresponsabilité de l’économie privée en politique de développement}\footnote{\fullcite{ulrichConceptPartenariatPublicprive2005a}}, Peter Ulrich et Florien Wettstein, deux chercheurs en science économique de l’Université de Saint-Gall, questionnent ce concept. Selon eux, des « présupposés normatifs » persistent : le secteur privé apporterait au secteur public cette efficacité qui lui manque nécessairement. Et ainsi, le privé serait indispensable aux institutions publiques qui ne pourraient pas avancer sans cela. A l’inverse, certains pourraient « diaboliser » le secteur privé, qui, seulement intéressé par la productivité, ne ferait pas bon ménage avec le rythme de la recherche. Pour autant, il semble évident qu’il est nécessaire de nuancer ces vues que l’on pourrait qualifier de caricaturales. L’exemple de SocFace est intéressant pour cela. L’initiative du projet est conjointe : en effet, Teklia et l’\gls{INED} travaillaient tous deux sur un projet similaire à présenter à l’ANR et ont décidés de s’associer avant d’en soumettre deux différents. On voit donc que l’initiative est partagée. 
On peut toutefois noter des différences, plus matérielles. Ainsi, l’\gls{INED} connait une gestion de la sécurité des serveurs plus stricte que ne peux le faire Teklia, qui est une plus petite structure et peut se permettre d’être plus souple. Par ailleurs, si le financement du projet se fait globalement par le biais de l’ANR, il faut noter que Teklia est une société privée, qui vend des solutions à des clients. Si l’objectif de l’\gls{INED} est bien la recherche, pour Teklia, le R\&D n’est qu’une partie de l’ensemble des activités.\\

Cette différence public/privé n’est donc pas une opposition mais une véritable complémentarité. Les équipes doivent collaborer. Mais n’appartenant pas aux mêmes secteurs de la recherche (les sciences sociales et la technologie), on peut parler d’une collaboration interdisciplinaire. A quel type d’interdisciplinarité se réfère-t-on ?

    \section{La collaboration des équipes, au-delà de l'interdisciplinarité}

Nous avons vu que les différents acteurs du projet SocFace étaient hétérogènes dans leur nature et leurs compétences et collaboraient pourtant. Plusieurs notions se recoupent pour définir le travail commun entre différentes institutions : interdisciplinarité, transdisciplinarité et pluridisciplinarité. A laquelle peut-on rattacher le projet SocFace ?\\

Le premier concept à être apparu est la pluridisciplinarité. 

\begin{quote} 
    \textit{"Elle peut être entendue comme une association de disciplines qui concourent à une réalisation commune, mais sans que chaque discipline ait à modifier sensiblement sa propre vision des choses et ses propres méthodes. À ce titre, la pluridisciplinarité existe depuis longtemps, même si son importance s'est accrue de nos jours."}\footnote{\fullcite{DefMulti}}
\end{quote}
Ce concept rassemble donc plusieurs disciplines, qui vont collaborer mais en restant chacune dans leur zone de compétence, sans réellement dialoguer entre elles. Les analyses des uns ne vont pas nécessairement interférer sur les résultats des autres. On pourrait comparer cette méthode de travail à la scolastique du Moyen-Âge, où les sept arts libéraux étaient associés à l’université.\\ 
L’interdisciplinarité est le degré supérieur de la collaboration, selon Pierre Delattre. Davantage qu’une simple discussion entre disciplines, on recherche ici à intégrer les concepts et les méthodes d’une autre discipline dans son travail. L’approche est plus générale, avec un entrecroisement des disciplines et une véritable interaction. Si les collaborations multidisciplinaires sont motivées par un sujet d’études commun, l’interdisciplinarité n’a pas seulement le sujet d’étude au centre de ses préoccupations : elle cherche à explorer les analyses, les perspectives, les résultats de chacune des disciplines concernées. Il n’y a pas de juxtaposition du travail de chacun, mais une véritable comparaison, et le travail des uns vient enrichir le travail des autres pour le faire avancer. Aujourd’hui, le domaine de la recherche est cloisonné : sciences dures ou sciences humaines, histoire ou géographie, médecine ou physique etc…  Pour autant, cette façon de travailler remonte à la nuit des temps : les philosophes grecs étaient astronome et mathématiciens, Léonard de Vinci concevait des armes de guerres et s’adonnait à la dissection, Vauban a construit des forteresses et écrit des traités de fiscalité. Le XX\textsuperscript{e} siècle, dans le sillage du siècle précédent, a vu les domaines de la recherche et de l’enseignement se démocratiser et s’élargir. Mais dans le même temps, les disciplines ont eu tendance à s’isoler les unes des autres, à se spécialiser. Selon Sandrine Louvel dans son article \textit{Ce que l’interdisciplinarité fait aux disciplines} \footnote{\fullcite{louvelCeQueInterdisciplinarite2015}}, la transition vers une meilleure coopération entre chercheurs commence vers la fin des années 60 et \textit{"trouve son apogée pendant les années 90"}. Cette impulsion est avant tout politique, et se traduit par le fait de favoriser les projets mettant en jeu plusieurs disciplines. Au-delà des disciplines, c’est le projet et les réponses apportées qui importent.\\ 
Enfin, la transdisciplinarité va encore plus loin, en cherchant à intégrer non seulement d’autres disciplines universitaires mais également des acteurs non académiciens. Le champ des collaborateurs travaillant sur un projet est élargi. Par exemple, on peut faire intervenir le personnel politique, une association, un groupe déterminé. Les théoriciens de la transdisciplinarité sont partis du constat que les problèmes contemporains, plus complexes selon eux, nécessitaient une réponse différente de ce que la recherche pouvait jusque là proposer.\\

De quelle notion relève SocFace? Selon les définitions données, on penche pour l’interdisciplinarité. De fait, la collaboration entre les différentes équipes du projet est étroite comme nous allons le voir. Par ailleurs, ces acteurs possèdent des compétences différentes : informaticien, ingénieur, historien, sociologue, démographe. Enfin, concernant la nature des acteurs : Teklia n’est pas un acteur académique au sens propre du terme. Dans un article sur l’interdisciplinarité dans les humanités numériques\footnote{\fullcite{benelQuelleInterdisciplinaritePour2014}}, Aurélien Bénel explique : 
\begin{quote}
\textit{"Ni l’usage d’un blog en histoire, ni la réalisation d’une base de données en littérature ne sont des projets interdisciplinaires. L’embauche d’un « informaticien » n’y change rien. L’interdisciplinarité nécessite un statut équivalent des représentants des disciplines, un respect de l’autre comme exerçant une discipline scientifique de plein droit. Quand une discipline est éclipsée par ses productions auprès du grand public, l’autre discipline ne doit pas l’enfermer dans ce qu’elle attend de ses productions, mais lui laisser la latitude d’innover. Ce besoin d’innovation signifie aussi qu’un type de projet, comme l’encodage savant de textes avec des balises, peut avoir été interdisciplinaire en 1987 et ne plus l’être aujourd’hui".}
\end{quote}
Cet article a été écrit en 2014 et illustre l’évolution de la notion d’interdisciplinarité mais aussi des humanités numériques jusqu’ici. Pour le projet SocFace, on ne parle pas d’"informaticiens" mais bien d’ingénieurs informatiques, de développeurs, d’architectes de la donnée. L’innovation – et donc la recherche - est au cœur du développement de l’entreprise. Les articles rédigés par les équipes de SocFace, destinés à des publications scientifiques, sont signés par des membres de Teklia, de l’\gls{INED},du \gls{SIAF} et de \gls{PSE}. Il y a une véritable reconnaissance de la compétence et de l’expertise de ces spécialistes du numériques. Cela ne veut pas dire pour autant que les chercheurs en sciences sociales du projet sont dépourvus de toute connaissance dans ce domaine. Ainsi, Lionel Kesztenbaum maitrise parfaitement les outils de gestion d’une base de données, et Christopher Kermorvan travaille depuis 10 ans avec des services d'archives. Cela permet un dialogue constant des équipes.\\

On voit donc que les acteurs du projet, bien que venant d’horizons différents, s’accordent autour d’un objectif commun. C’est la définition de l’interdisciplinarité : la collaboration entre plusieurs disciplines différentes et complémentaires. Si la collaboration ne se fait pas toujours dans l’interdisciplinarité, c’est bien le cas ici. C’est pourquoi il faut trouver des façons de travailler ensemble et se répartir les différentes étapes du projet, selon les compétences de chacun. 