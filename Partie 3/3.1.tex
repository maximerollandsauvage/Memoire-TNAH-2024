\chapter{Le meilleur des deux mondes pour affronter des données d’ampleur}

Les objectifs ambitieux de ce projet nécessitent une collaboration étroite entre des acteurs aux compétences complémentaires, combinant la technologie avancée des entreprises avec la rigueur scientifique des institutions publiques. Cette collaboration ne se limite pas à une simple nécessité pratique mais s’inscrit dans un contexte historique et socio-économique où les collaboration entre public et privé sont devenus une réponse à la diminution des financements publics. Dans ce cadre, SocFace illustre les bénéfices de cette coopération pour mener à bien des projets d’envergure. Nous examinerons d’abord comment cette alliance s’est développée, avant d’analyser les avantages pour les entreprises privées, puis de discuter des limites des institutions publiques dans ce type de collaboration. 

    \section{La nécessité des uns pour les autres, travailler ensemble}

La littérature centrée sur la recherche partenariale commencent toujours par nous rappeler le contexte dans lequel se sont développés ces partenariats. Jusque dans les années 80, la recherche publique restait le plus souvent initiée, opérée et financée par des agents publics. Cette politique est le fruit d’une longue institutionnalisation des centres de recherches et universités françaises. Mais avec le développement de la mondialisation, le mouvement de privatisation que connaît une partie des pays développés, le budget alloué à la recherche a baissé. On peut prendre les chiffres avancés par le rapport de la mission sur l’écosystème de la recherche et de l’innovation\footnote{\fullcite{rapport_mission}}. Il avance dans un premier temps que les investissements de l’ANR dans la loi pour la recherche de 2030 sont en augmentation : 

\begin{figure}[H]
        \centering
        \includegraphics[width=1.0\linewidth]{Figures/Partie 3/Fig.3.1 - Schéma ANR.png}
        \caption{Budget d’intervention hors investissements d’avenir/France 2030 et mesures de préservation de l’emploi dans la R\&D privée prévues dans le plan de relance.\\
        \textit{Crédit : ANR}}
        \label{fig:Fig3.1}
    \end{figure}

Cela se traduit par des effets positifs : le taux de réussite des appels à projets de l’ANR en 2022 est de 24\%, alors qu’il ne dépassait pas les 10\% en 2010. Mais ces chiffres ne doivent pas masquer une autre réalité : le budget global alloué à la recherche ne dépasse pas l’objectif de 3\% fixé par les différents gouvernements depuis des décennies. C’est un des plus bas des pays de l’OCDE selon le rapport. Ainsi, les budgets de fonctionnement des laboratoires et centres de recherches sont en baisses constantes. Par exemple, le rapport d’Auto-Evaluation du CNRS décrit ainsi son budget au cours des dix dernières années : 

\begin{quote}
\textit{"Entre 2012 et 2021, le CNRS a perdu 4,3\% de ses effectifs rémunérés sur la subvention pour charges de service public (24 685 contre 25 787) alors que dans le même temps, la part de cette subvention consacrée aux dépenses de personnel est passée de 82,2\% à 84,1\%. Mécaniquement, le pourcentage de la subvention disponible pour le fonctionnement et les investissements a diminué de 2\%, passant de 17,4\% à 15,4\%. Cette « double peine »  moins de personnel et un budget de fonctionnement et d'investissement plus faible  a manifestement réduit la capacité de l'organisme à développer et à mettre en œuvre une véritable politique scientifique".}  
\end{quote}

Avec ce genre de politique, on comprend que la recherche partenariale soient encouragés. On pourrait presque penser que cela permet aux pouvoirs publics de se désengager de la recherche. Pour autant, concernant le projet SocFace, la situation est plus complexe : il n’y a pas eu de financement privé, dans le sens où aucune entreprise n’a fourni d’argent pour le projet. Mais l’un des porteurs principal du projet est une entreprise privée et fourni dès lors sa propre technologie, son expertise, ses locaux et surtout sa main d’œuvre. C'est un apport en nature. Il y a donc un soutien privé, qui pallie le manque de budget de la recherche considéré à plus large échelle. De fait, on peut supposer que si la recherche, depuis tant d’années, avait été mieux financée, la technologie développée par Teklia aurait pu été développée par l’INED directement. Bien entendu, cela reste une supposition, qui ne peut absolument pas être vérifiée.\\ 

SocFace semble toutefois indiquer que le recours à la recherche privée devient une nécessité pour gérer ce type de projet. Reste à trouver l’entreprise qui correspond à ce besoin, qui doit également y trouver son intérêt. 

\section{De l'intérêt d'entrer dans la recherche pour Teklia}

Quel est intérêt de participer à projet de recherche pour Teklia, entreprise spécialisée dans le traitement automatique des documents et des données massives et qui est déjà à la pointe de l’expertise en intelligence artificielle, en vision par ordinateur et en traitement du langage naturel. \\

D’une part, on peut penser que ce partenariat participe du mouvement initié par les pouvoirs publics depuis quelques années en matière d’incitation à la recherche. La politique fiscale tente en effet de favoriser la participation du secteur privé à l’innovation, motivant les start-ups avec des crédits d’impôts. Teklia ayant été créé en 2014, on ne peut plus vraiment la considérer comme une start-up mais elle est désormais une entreprise installée et innovante. Ainsi, on a le Crédit d'Impôt Recherche (CIR) qui permet aux entreprises de bénéficier d’un crédit d’impôt équivalent à 30\% des dépenses de recherche et développement (R\&D) jusqu'à 100 millions d'euros, et 5\% au-delà. Le CIR couvre les dépenses liées aux salaires des chercheurs, à l'achat de matériel, aux frais de sous-traitance de la recherche etc.... Cela réduit considérablement le coût net des projets de recherche pour l’entreprise. Les chiffres les plus récents datent de 2021 et font état d’un montant de 7,25 milliards d’euros de crédit. On doit donc que c’est une incitation qui fonctionne. \\
D’autre part, on a vu dans l’explication du déroulement du projet que la technologie développée par Teklia nécessitait une forte puissance de calcul, puisque l’ensemble des opérations sont faites en une seule passe. C’est pourquoi SocFace utilise Jean Zay, du CNRS. On peut penser que sans sa participation à un projet de recherche publique, Teklia n’aurait pas eu accès à un outil aussi puissant et aussi sophistiqué. Aussi, il est intéressant pour Teklia de tester sa technologie sur un corpus aussi énorme que les listes de recensement. C’est un défi de taille qui ne peut que permettre d’améliorer leurs outils et affiner leur technologie. Il y a donc aussi un intérêt pour leur politique de développement.\\
Enfin, Teklia est une entreprise, qui produit des solutions pour des clients. Le fait de participer à un projet de recherche publique est une très bonne carte de visite auprès de potentiels clients d’institutions publiques. Ce genre de marché est intéressant pour eux puisqu’ils sont spécialisés dans la lecture de manuscrits, généralement possédés par des institutions du patrimoine, généralement publiques. SocFace est un projet innovant, qui peut se décliner sur beaucoup de corpus différents. Cela ouvre des perspectives commerciales importantes pour Teklia.\\

Ainsi, on voit que les bénéfices pour une entreprise privée sont aussi très intéressants, tant financiers que dans leurs perspectives de développement technologique ou commerciale. C’est un échange de bons procédés entre institutions privées et publiques.  

\section{Les limites des institutions publiques en matière d'humanités numériques?}

Le fait que SocFace ait eu besoin de recourir à la sphère privée veut-il dire pour autant que l’intervention de la sphère publique en matière d’humanités numériques est limitée ? La réponse n’est pas évidente mais la question mérite d’être posée.\\

Il faut commencer par préciser que la définition des "humanités numériques" est assez large. On peut reprendre la définition donnée par Marie-Laure Massot et Agnès Tricoche, dans un article exposant leurs conclusions sur l’évolution des humanités numériques à l’ENS entre 2017 et 2018\footnote{\fullcite{massotRenewalDigitalHumanities2021}} : 

\begin{quote}
C'est l'exploitation de l'outil numérique dans le processus intellectuel de production et d'analyse des humanités et des sciences sociales, une alliance ayant pour objectif d’optimiser le potentiel d’exploitation et de valorisation des données scientifiques."
\end{quote}   

C’est une définition large et flexible, qui s’adapte à un objet toujours plus mouvant. Il est donc fort probable que ce soit cette absence de définition et de périmètre clair qui empêche la puissance publique de pouvoir favoriser et structurer efficacement les humanités numériques. Il n’y a pas à proprement parler de département "humanités numériques" dans la plupart des universités ou organismes de recherches. La matière "humanité numérique" est distillée dans toutes les disciplines. On pourrait donc trouver ici une limite : les humanités numériques, si elles sont reconnues dans la recherche, ne sont pas encore considérée comme une discipline à part entière et sont maintenues dans un système hybride et le plus souvent rattachées à une discipline académique. 
	
Toutefois, on doit préciser qu’il existe des infrastructures orientées vers les humanités numériques. Huma-Num a justement été crée pour favoriser l’accès, le partage, l’exploitation et la conservation des données numériques produites par la recherche en sciences sociale. Créée en 2013, elle est placée sous la tutelle du CNRS et possède des antennes en région, avec les Maisons des Sciences Humaines. Son objectif est de pouvoir faciliter l’accès aux outils numériques pour des laboratoires, projets de recherches ou chercheurs : stockage des données, gestion des formats, création de bases de données etc… Huma-Num favorise également les discussions européennes et encourage les projets transeuropéens. Pour autant, la structure ne bénéficie pas d’un gros budget : 80 000€ en 2019 et seulement 27 salariés (permanents et non permanents). Cette structure permet davantage d’aider un projet que de lui fournir une véritable technologie. Il permet de créer des consortiums, de faire se rencontrer des équipes pouvant collaborer. On peut penser que c’est pour cela que SocFace n’a pas sollicité l’aide d’Huma-Num. L’ampleur du projet n’était pas compatible avec les capacités d’Huma-Num. Il existe par ailleurs des laboratoires spécialisés en \gls{océrisation}\footnote{Voir Glossaire} et en \gls{HTR}, notamment le LITIS, qui dépend de l’Université de Rouen, et qui mène des recherches poussées en intelligence artificielle. Mais ils sont peu en France et développent leurs propres projets. Ainsi, le LITIS est porteur du projet POPP que l’on a mentionné plus haut.\\
	
On ne peut donc pas dire que les humanités numériques en France ne sont pas aidées, ou invisibles à la recherche. Mais elles ne le sont sans doute pas assez pour permettre un développement à 100\% public de tous les projets. D’où la structuration de SocFace, avec une partie issue de la recherche privée. 



