\chapter{L'interdisciplinarité au coeur de l'avenir de la recherche}

L'originalité de SocFace ne réside pas dans son interdisciplinarité. La majorité des projets aidés par l’ANR font intervenir plusieurs disciplines, à divers degrés. De même, la recherche partenariale est de plus en plus valorisés, comme en témoigne la dernière publication de l'ANR\footnote{\fullcite{anr2022}}, qui accorde une attention particulière à ces partenariats. Tout cela démontre que, loin d’être singulier, SocFace s’inscrit dans le mouvement de la science ouverte, qui irrigue le monde de la recherche depuis quelques décennies. Nous examinerons d'abord les éléments du projet qui le démontrent, avant d'en analyser les implications, notamment en ce qui concerne la définition du métier de chercheur.

    \section{S'inscrire dans le mouvement de la science ouverte}

Depuis l’antiquité, la communauté scientifique cherche à partager son savoir avec l’humanité. De même, les scientifiques se sont toujours inspirés les uns les autres, dans un dialogue constant qui a permis les plus grandes découvertes. Le travail scientifique est – depuis toujours – basé sur le partage des découvertes et les grandes avancées sont issues de collaboration, de dialogues entre scientifiques, philosophes, mathématiciens etc… Le mouvement de la science ouverte, tel qu’on l’entend aujourd’hui, est le produit des avancées technologiques récentes qui permettent un accès plus simple et plus immédiat aux découvertes scientifiques. Ce mouvement a donc été théorisé : 

\begin{figure}[H]
        \centering
        \includegraphics[width=1.0\linewidth]{Figures/Partie 3/Fig.3.2 - Schéma Science ouverte.png}
        \caption{Composantes de la science ouverte.\\
        \textit{Crédit : Université de Montpellier}}
        \label{fig:Fig3.2}
    \end{figure}

On retrouve six principes clefs : 

\begin{enumerate}
    \item \textbf{L’accès ouvert} : rendre disponibles les résultats de recherches de façon libre et accessible
    \item \textbf{Les données ouvertes} : rendre les données disponibles, réutilisables et accompagnées de leurs métadonnées pour faciliter leur reproduction. On parle du principe FAIR, Findable (= trouvable), Accessible, Interoperable, Reusable (=réutilisable)
    \item \textbf{Logiciels et codes ouverts} :  on parle ici des codes sources développés dans le cadre du projet de recherche, pour mieux comprendre la méthodologie de la recherche
    \item \textbf{Evaluation ouverte par les pairs} : cela permet d’assurer la transparence des modes d’évaluations et permettre à la communauté scientifique de participer à l’évaluation des travaux de leurs collègues
    \item \textbf{Carnets de recherches ouverts} : on parle des journaux de recherches, afin de mieux saisir les doutes, questionnements, hypothèses que l’équipe de recherche a développé tout au long du projet 
    \item \textbf{Science participative} : faire participer le grand public à des projets de recherche
\end{enumerate}

Depuis vingt ans, la science ouverte est au cœur des préoccupations des organismes de recherche. On favorise l’accès libre aux publications (Cairn, HAL), on pousse les chercheurs à collaborer, on favorise le dialogue entre discipline etc… Nous n’allons pas ici exposer les avantages et les inconvénients de la science ouverte car il est désormais acquis que la science ouverte est bénéfique au monde scientifique et à la recherche. La très grande majorité des projets actuels, particulièrement en sciences humaines, s’inscrivent dans ce mouvement. Il resterait à explorer les questions juridiques, et particulièrement du droit d’auteur, qui continuent de poser problème dans un mouvement qui prône un accès totalement libre. Mais ce n’est pas l’objet de notre propos. \\
Comment SocFace adopte cette science ouverte - au delà de l'interdisciplinarité déjà démontrée?  Il faut commencer par l’objectif final du projet : mettre à la disposition du grand public une base de données de grandes ampleur. C’est la définition même de la science ouverte : l’open access pour tous des résultats d’un projet de recherche. Bien évidemment, en l’espèce les données des listes de recensements, c’est-à-dire les noms et les adresses étaient déjà accessibles. Mais le résultat du projet est la construction de la base de données elle-même et c’est ce qui est laissé en libre accès. Concernant la construction du projet, et particulièrement des outils permettant l’HTR, Teklia s’est basé sur un modèle existant déjà, \DAN{}. Celui-ci a été laissé en libre accès par son créateur Denis Coqueret, comme on l’a vu dans la deuxième partie. Par ailleurs, les équipes ont récupérés certaines données du projet Popp, dont l’objectif est sensiblement identique à SocFace mais limité à Paris sur l’entre-deux guerres. C’est donc un dialogue entre deux projets de recherche, exactement ce qui est encouragé par la science ouverte. Enfin, on a expliqué que les modèles étaient entraînés par des annotateurs. Ces équipes d’annotations sont parfois des spécialistes en généalogies, le plus souvent des étudiants rémunérés. Dans tous les cas ces personnes habituées à ce genre de campagne. Mais ils ne sont pas chercheurs ou n’appartiennent pas à la recherche académique. SocFace s’inscrit donc dans la science participative.\\

SocFace est un bon exemple de projet de la science ouverte, ce qui démontre comment le projet est un projet de son temps. Il s’inscrit dans un contexte général d’ouverture de la recherche à d’autres domaines. Ce qui n’est pas surprenant pour un projet qui place la collaboration et l’interdisciplinarité au cœur de son système de fonctionnement. 

    \section{Vers une évolution du statut du chercheur?}

Ce sont les interactions entre chercheurs et techniciens qui sont au cœur de ce projet et qui en font l’architecture. Cette interdisciplinarité n'est pas nouvelle, et Louis Henry l'évoquait déjà dans un article de 1968\footnote{\fullcite{henry}}. Selon lui, les \textit{hommes de l'art} peuvent être des ingénieurs, des chercheurs ou des industriels. Les choses ont-elles évoluées en 50 ans? \\
Pour commencer, on peut dire qu'elle est désormais recherchée et valorisée. Ce dialogue n'est bien évidemment pas apparu avec le numérique, et déjà en 1994\footnote{\fullcite{morinInterdisciplinarite1994}}, Edgar Morin parlait de la \textit{"polycompétence du chercheur"} en prenant l’exemple des préhistoriens qui doivent, en plus de leur propre expertise, maîtriser l’écologie, la psychologie, l’éthologie – entre autres. Un chercheur doit donc, de façon habituelle, sortir de son domaine de spécialisation. Pour autant, si l'interdisciplinarité est encouragée, elle mériterait d'être davantage considérée comme un sujet d'études en elle-même dans les projets de recherche. Ainsi, dans un article paru en 2006\footnote{\fullcite{buhlerDossierInterdisciplinariteJeune2006}}, Eve-Anne Bühler  évoque le fait que le jeune chercheur pratique l’interdisciplinarité de façon intuitive, sans y réfléchir. Ce qui n’est pas une bonne solution selon l’auteure : 

\begin{quote}
    \textit{"Trop souvent, le recours à d’autres disciplines se fait inconsciemment, de sorte que tout le questionnement autour de la démarche elle-même est occulté du travail"}
\end{quote}. 

Mais l’interdisciplinarité entre sciences sociales et numériques implique des méthodes de travail et une appréhension d’une technologie différente. Là où on peut trouver des points communs dans la façon de travailler entre un chercheur en histoire et un sociologue ou un démographe, le rapprochement est plus délicat avec un ingénieur informatique ou un spécialiste des données. Il y aura donc une plus grande réflexion sur la démarche interdisciplinaire, justement parce que cette démarche demandera un effort supplémentaire pour le chercheur en sciences « académiques. Pour autant, on peut penser qu’avec le développement des nouvelles technologies, et l’intégration toujours grandissante du digital dans la vie quotidienne, il est fort probable que les chercheurs des temps futurs auront moins ces difficultés d’adaptation et que ce type d’interdisciplinarité et de manière de collaborer leur paraîtront évidente. \\
L’interdisciplinarité entre numérique et sciences sociales peut également faire évoluer le statut de l’ingénieur informatique. Celui-ci n’est pas, en principe, un habitué de la recherche, au sens universitaire du terme. Les entreprises spécialisées dans l’innovation ont bien entendu la plupart du temps des départements R\&D, mais elles rencontrent rarement le milieu de la recherche universitaire. Avec l’intégration toujours plus grande du digital, voire de l’intelligence artificielle dans la recherche, il est certain que les collaborations entre ces institutions deviendront de plus en plus courantes. Peut-on dès lors considérer les ingénieurs informatiques comme des chercheurs ? Il n’y a pas de définition légale du chercheur. Ce terme est employé par les instituts de recherche qui qualifient ainsi une partie de leur personnel. La seule définition légale existante concerne les enseignants-chercheurs, dans le code de l’éducation. Doit-on dès lors penser qu’un chercheur doit forcément être employé par un institut de recherche pour se définir comme tel ? Le projet SocFace nous démontre le contraire : Christopher Kermorvan, ou Bastien Abadie effectuent toute la recherche sur la partie technologique du projet. Ils participent aux conférences qui présentent le projet, ils cosignent les articles sur le projet, publiés dans des revues scientifiques. On peut dès lors les qualifier de chercheurs. Cela ouvre une perspective intéressante de la recherche, qui reste parfois cloisonnée sur les mêmes institutions, et sur \textit{cursus honorum} attendus. Les compétences de ces experts vont devenir de plus en plus essentielles aux projets de recherche. Et c’est cette nécessité qui leur apporte une nouvelle légitimité. \\
L’interdisciplinarité telle qu’on la voit dans le projet SocFace, c’est-à-dire entre sciences sociales et numérique permet donc d’ouvrir le champ de la recherche. En élargissant la collaboration entre disciplines, on se dirige vers une redéfinition du statut du chercheur, ce qui pourrait modifier considérablement l’élaboration d’un projet de recherche, son financement et sa mise en œuvre. 
